\documentclass{aastex61}

% For revision history
\IfFileExists{vc.tex}{\input{vc.tex}}{
    \newcommand{\githash}{UNKNOWN}
    \newcommand{\giturl}{UNKNOWN}}

% Define commands
\newcommand{\acronym}[1]{{\small{#1}}}
\newcommand{\project}[1]{\textsl{#1}}

\received{}
\revised{}
\accepted{}

%\submitjournal{AAS Journals}

\shorttitle{Chemical heterogeneity in wide binary systems}
\shortauthors{Casey \& Bovy}

\begin{document}


\title{Chemical heterogeneity in wide binary systems}


\author[0000-0003-0174-0564]{Andrew R. Casey}
\email{arc@ast.cam.ac.uk}
\affil{Institute of Astronomy, Madingley Road, Cambridge CB3~0HA, UK}

\author[0000-0001-6855-442X]{Jo Bovy}
\email{bovy@astro.utoronto.ca}
\affil{
    Department of Astronomy and Astrophysics, University of Toronto, 
    50 St. George Street, Toronto, ON, M5S 3H4, Canada}
\affil{
    Center for Computational Astrophysics, 162 5th Ave, New York, NY 10010, USA}


\begin{abstract}
This example manuscript is intended to serve as a tutorial and template for
authors to use when writing their own AAS Journal articles. The manuscript
includes a history of aastex and documents the new features in the
previous version, 6.0, as well as the new features in version 6.1. This
manuscript includes many figure and table examples to illustrate these new
features.  Information on features not explicitly mentioned in the article
can be viewed in the manuscript comments or more extensive online
documentation. Authors are welcome replace the text, tables, figures, and
bibliography with their own and submit the resulting manuscript to the AAS
Journals peer review system.  The first lesson in the tutorial is to remind
authors that the AAS Journals, the Astrophysical Journal (ApJ), the
Astrophysical Journal Letters (ApJL), and Astronomical Journal (AJ), all
have a 250 word limit for the abstract.  If you exceed this length the
Editorial office will ask you to shorten it.
\end{abstract}

%\keywords{}

\section{Introduction} 
\label{sec:introduction}
Write.


\acknowledgments

ack


%\software{astropy, test} 

\begin{thebibliography}{}

\bibitem[Astropy Collaboration et al.(2013)]{2013AA...558A..33A} Astropy Collaboration, Robitaille, T.~P., Tollerud, E.~J., et al.\ 2013, \aap, 558, A33 
\bibitem[Vogt et al.(2014)]{2014ApJ...793..127V} Vogt, F.~P.~A., Dopita, M.~A., Kewley, L.~J., et al.\ 2014, \apj, 793, 127  

\end{thebibliography}

\end{document}
